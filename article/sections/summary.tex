\section{Summary}

This paper presented a graph-based framework for coverage analysis in autonomous driving that represents traffic scenes as hierarchical graphs combining map topology with actor relationships.
The framework employs a two-phase graph construction algorithm to systematically capture spatial relationships between traffic participants, encoding leading, following, neighboring, and opposing configurations as typed edges in a directed multigraph structure.

Two complementary approaches for coverage analysis were developed and evaluated.
The subgraph isomorphism approach utilizes a set of 18 manually defined archetype graphs representing common driving scenarios, achieving 62\% node coverage on the evaluated datasets.
This method provides interpretable coverage metrics by identifying which specific traffic patterns are present in a given scene.
The graph embedding approach leverages Graph Isomorphism Networks with Edge features trained via self-supervised contrastive learning to project traffic scenes into a continuous vector space.
This representation enables similarity-based coverage assessment, clustering of related scenarios, and anomaly detection for identifying underrepresented traffic situations.

Both approaches were validated on real-world data from the Argoverse 2.0 dataset and synthetic data from the CARLA simulator, as presented in Section~\ref{chapter:application}.
The experimental results demonstrate that the framework successfully captures meaningful traffic scene structure and reveals notable differences in scenario distributions between the two datasets.

The proposed approach offers several advantages over traditional coverage analysis methods.
Most significantly, it scales efficiently to diverse traffic scenarios without requiring scenario-specific handling rules.
In contrast to approaches such as TNO Streetwise \cite{tno_streetwise}, where each scenario type must be manually defined and processed individually, the presented framework only requires the definition of actor graph construction rules and, optionally, archetype graphs.
Furthermore, the approach naturally accommodates varying numbers of actors in a scene without excluding participants or focusing on arbitrary subsets.

Future work will extend the framework in two directions.
First, the actor graph representation will be enhanced to incorporate temporal information by including multiple time steps within a single graph structure.
Second, methods for automatically extracting archetype graphs from real-world traffic observation data will be investigated, reducing the reliance on manual archetype definition.

The source code for this research is publicly available at \url{https://github.com/tmuehlen80/graph_coverage}.

