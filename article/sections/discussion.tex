\section{Application}
\label{chapter:application}

\subsection{Argoverse 2.0}

\cite{Argoverse2}

%TODO MARIUS: Describe the Argoverse 2.0 dataset.

\subsection{Carla}

CARLA (Car Learning to Act, \cite{Dosovitskiy17Carla}) is an open-source simulator specifically designed for autonomous driving research and 
development. It provides a highly realistic urban driving environment with 
diverse road layouts, weather conditions, and traffic scenarios. The simulator features a comprehensive 
sensor suite simulation, flexible API for scenario creation, and supports both learning-based 
and traditional autonomous driving approaches. CARLA enables researchers to test and 
validate autonomous vehicle systems in a safe, controllable environment before real-world deployment.

The simulator has gained widespread adoption across both academic and industrial settings. In research, CARLA serves as a standard platform for developing and 
benchmarking autonomous driving algorithms, including reinforcement learning approaches for vehicle control and sensor fusion 
techniques \cite{codevilla2019exploringlimitationsbehaviorcloning}. Industry applications include 
virtual testing of production autonomous vehicle systems, scenario-based validation pipelines, and integration 
with hardware-in-the-loop testing frameworks \cite{jaeger2023hiddenbiasesendtoenddriving}. CARLA 
is also extensively used in autonomous driving competitions and challenges, providing a common evaluation 
environment for comparing different approaches across research groups worldwide.


Here, Carla version $0.9.15$ is used. The CARLA version $0.10.0$ is not used, because it had only 2 maps and
Mine\_1 (which is not really normal roads) at the start of this project.
Specifically, the following maps were used: Town01, Town02, Town03, Town04, Town05 and Town07. Plots
of these maps are shown in Figure \ref{fig:carla_maps}.

\begin{figure}[h]
\centering
\includegraphics[width=0.8\textwidth]{plots/carla_maps_used.png}
\caption{Overview of CARLA maps used in the simulation study: Town01, Town02, Town03, Town04, Town05, and Town07. These maps provide diverse urban driving environments with varying road layouts, intersections, and traffic patterns.}
\label{fig:carla_maps}
\end{figure}

The data generation script implements sophisticated behavior control mechanisms to create 
diverse and realistic traffic scenarios. Multiple vehicle types including trucks, motorcycles, 
and regular cars are spawned with varying probabilities, each exhibiting different behavioral
characteristics such as speed preferences, following distances, and lane-changing tendencies. 
The script incorporates dynamic behavior modifications during simulation, including random slowdowns, 
periodic behavior changes, and adaptive responses to traffic conditions, resulting in rich and 
varied traffic scene data across multiple CARLA maps and simulation iterations. The simulation runs
have between 20 and 60 vehicles each.

The resulting data consists of xxx scenes with 11 seconds of simulation time each, in order to have a 
similar data size as the Argoverse 2.0 dataset.


\subsection{Subgraph Isomorphism Coverage Analysis}

We apply the subgraph isomorphism approach to both the  CARLA and the Argoverse data to identify coverage scenarios.
The archetypes used here are defined manually and described in table xxx. The chosen archetypes are typical traffic situations like e.g. lead vehicle situations, lane change situations or different combinations of opposite direction vehicles.
Also, information like which traffic actors are on an intersection are incorporated into the archetypes.
The results are shown in Figure \ref{fig:subgraph_isomorphism_coverage_barcharts} to Figure \ref{fig:combined_distributions_plots_speed_distance_argo}.
In Figure \ref{fig:subgraph_isomorphism_coverage_barcharts} there is a clear signal that for both datasets the coverage is not uniform per the different archetypes.
Also, the distribution for the Carla dataset is not close to the Argoverse distribution, indicating that the CARLA data is not representative for the Argoverse data.
Even worse, the Carla dataset is nearly completely missing out e.g. on the cut\_out\_intersection archetype, clearly indicating a coverage gap in the Carla dataset.
A next step of analysis is to check, which archetypes are occuring simultaneously in a traffic scene. 
Figure \ref{fig:subgraph_isomorphism_agreement_matrix_manual_scenarios} shows the agreement matrix for the manually defined coverage scenarios for CARLA and Argoverse.
The heatmaps show the percentage of agreement between the manually defined coverage scenarios for CARLA and Argoverse.

% TODO: Write a sentence about the agreement matrices once using the actual actor graph defintion.

A last example of how to use the assignment of archetypes to traffic scenes is to check the parameter distribution for the speed and path length of the traffic scenes.
Figure \ref{fig:combined_distributions_plots_speed_distance_carla} and Figure \ref{fig:combined_distributions_plots_speed_distance_argo} show the parameter distribution for the speed and path length of the traffic scenes for CARLA and Argoverse.
The plots show that the distribution for the CARLA data is not close to the Argoverse distribution, indicating that the CARLA data is not representative for the Argoverse data.
Given the assignment of archetypes to traffic scenes based on the actors can be used in further analysis not done here, e.g. to check more divers parameters and distributions.


\begin{figure}[h]
    \centering
    \begin{subfigure}[b]{0.48\textwidth}
        \centering
        \includegraphics[width=\textwidth]{plots/subgraph_isomorphism_carla_coverage_barchart.png}
        \caption{CARLA coverage scenarios}
        \label{fig:subgraph_isomorphism_carla_coverage_barchart}
    \end{subfigure}
    \hfill
    \begin{subfigure}[b]{0.48\textwidth}
        \centering
        \includegraphics[width=\textwidth]{plots/subgraph_isomorphism_argo_coverage_barchart.png}
        \caption{Argoverse coverage scenarios}
        \label{fig:subgraph_isomorphism_argo_coverage_barchart}
    \end{subfigure}
    \caption{Coverage barcharts for the manually defined coverage scenarios for CARLA and Argoverse.}
    \label{fig:subgraph_isomorphism_coverage_barcharts}
\end{figure}

% \begin{figure}[h]
%     \centering
%     \includegraphics[width=0.8\textwidth]{plots/subgraph_isomorphism_agreement_matrix_manual_scenarios_carla.png}
%     \caption{Agreement matrix for manually defined coverage scenarios for CARLA.}
%     \label{fig:subgraph_isomorphism_agreement_matrix_manual_scenarios_carla}
% \end{figure}


\begin{figure}[h]
    \centering
    \begin{subfigure}[b]{0.48\textwidth}
        \centering
        \includegraphics[width=\textwidth]{plots/subgraph_isomorphism_agreement_matrix_manual_scenarios_carla.png}
        \caption{Agreement matrix for CARLA.}  
        \label{fig:subgraph_isomorphism_agreement_matrix_manual_scenarios_carla}
    \end{subfigure}
    \hfill
    \begin{subfigure}[b]{0.48\textwidth}
        \centering
        \includegraphics[width=\textwidth]{plots/subgraph_isomorphism_agreement_matrix_manual_scenarios_argo.png}
        \caption{Agreement matrix for Argoverse.}  
        \label{fig:subgraph_isomorphism_agreement_matrix_manual_scenarios_argo}
    \end{subfigure}
    \caption{Agreement matrix for manually defined coverage scenarios for CARLA and Argoverse.}
    \label{fig:subgraph_isomorphism_agreement_matrix_manual_scenarios}
\end{figure}

% \begin{table}[htbp]
%     \caption{My Table}
%     \label{tab:mytable}
%     \begin{tabular}{rrrrr}
%     \toprule
%     scn 1 & scn 2 & scn 3 & scn 4 & size \\
%     \midrule
%     False & False & False & False & 613 \\
%     False & False & True & False & 3 \\
%     True & True & False & True & 377 \\
%     True & True & True & True & 7 \\
%     \bottomrule
%     \end{tabular}
% \end{table}
% % \cite{goodfellow2016deep} % needed?

\begin{figure}[h]
    \centering
    \includegraphics[width=0.85\textwidth]{plots/combined_distributions_plots_speed_distance_carla.png}
    \caption{Parameter distribution for speed and path length for CARLA.}
    \label{fig:combined_distributions_plots_speed_distance_carla}
\end{figure}

\begin{figure}[h]
    \centering
    \includegraphics[width=0.85\textwidth]{plots/combined_distributions_plots_speed_distance_argo.png}
    \caption{Parameter distribution for speed and path length for Argoverse.}
    \label{fig:combined_distributions_plots_speed_distance_argo}
\end{figure}

\subsection{Graph Embeddings Coverage Analysis}

\begin{figure}[h]
    \centering
    \includegraphics[width=0.8\textwidth]{plots/train_test_graph_embeddings_loss_plot.png}
    \caption{Training and test loss for the graph embeddings model trained jointly on CARLA and Argoverse 2.0 data.}
    \label{fig:train_test_graph_embeddings_loss_plot}
\end{figure}
    

The resulting embeddings have been analysed in a number of ways. For plausibility checks, for a number of 
randomly samples scenarios, the scenario with 
closest embedding vector (euclidean distance) has been visualized. 
This is shown in Figure \ref{fig:plausibility_checks}. This 
includes comparison between CARLA and Argoverse scenarios.

\begin{figure}[h]
    \centering
    \includegraphics[width=0.8\textwidth]{plots/graph_embeddings_pca_tsne_test_plot.png}
    \caption{PCA and t-SNE visualization of the embedding space for Carla and Argoverse 2.0 scenarios.}
    \label{fig:embedding_space}
\end{figure}
    
As a next step, the embedding space has been analysed using PCA and t-SNE. This is shown in Figure \ref{fig:embedding_space}. This can be done in a number of 
different flavors, like for example:

\begin{itemize}
    \item distinguishing between CARLA and Argoverse scenarios by a color coding, 
    \item visualiizing just the CARLA scenarios and color code them by the map, in order to see if the maps deliver different types of scenarios,
    \item by calculating a density distribution of the embedding space for the Argoverse scenarios, and to check if for a regions with a density 
    about a certain threshold, there exist CARLA scenarios, i.e. do a coverage analysis in the embedded space
    \item do the same vice versa, i.e. to check for relevance of CARLA scenarios.
\end{itemize}

As the embedding space is a normal metric space, representing scenarios across a large range of different driving situations, these embeddings can be used also for 
many other tasks, like for example clustering, anomaly detection, similarity search, etc.

And, as the main task considered here is coverage analysis, the embeddings can be used to check if a target distribution is met by a test distribution in the embedded space. 
Specifically, considering the Argoverse 2.0 scenarios as the target distribution, and the CARLA scenarios as the test distribution, the embeddings can be used to check 
if the CARLA scenarios cover the Argoverse 2.0 scenarios in the embedded space.


\begin{figure}[h]
    \centering
    \includegraphics[width=0.95\textwidth]{plots/graph_embeddings_comparison_plausibility_check_0.png}
    
    \includegraphics[width=0.95\textwidth]{plots/graph_embeddings_comparison_plausibility_check_1.png}
    
    \includegraphics[width=0.95\textwidth]{plots/graph_embeddings_comparison_plausibility_check_2.png}
    
    \includegraphics[width=0.95\textwidth]{plots/graph_embeddings_comparison_plausibility_check_3.png}
    
    \includegraphics[width=0.95\textwidth]{plots/graph_embeddings_comparison_plausibility_check_4.png}
    \caption{Plausibility checks showing graph embedding comparisons. For randomly sampled scenarios, the most similar scenarios based on embedding distance (both Euclidean and cosine) are visualized, including comparisons between CARLA and Argoverse scenarios.}
    \label{fig:plausibility_checks}
\end{figure}

