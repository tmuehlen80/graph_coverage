\section{Defining a traffic scene graph}
\label{chapter:defining_a_traffic_scene_graph}


\subsection{Datasets}
In the following, 2 datasets used for developing and demonstrating the suggested methods.

\textbf{Argoverse 2.0} \cite{Argoverse2} is a large-scale dataset for autonomous driving research,
developed by Argo AI.  It contains a Dataset of 250,000 scenarios of 11 seconds each, featuring
tracked object trajectories for vehicles, pedestrians, cyclists, and other road users.

The dataset contains object classifications, track identities across frames, and semantic map
information including lane boundaries, crosswalks, and traffic signal locations across diverse
geographic locations. A subset of 17000 scenes from the train partition is used for the analysis.

\textbf{CARLA} (Car Learning to Act, \cite{Dosovitskiy17Carla}) is an open-source simulator specifically designed for autonomous driving research and 
development. It provides a realistic urban driving environment with 
diverse road layouts, weather conditions, and traffic scenarios.

Here, Carla version $0.9.15$ is used, because it provides more maps than subsequent versions.
Specifically, the following maps were used: Town01, Town02, Town03, Town04, Town05 and Town07. 
% Plots of these maps are shown in Figure \ref{fig:carla_maps}.

% \begin{figure}[ht]
% \centering
% \includegraphics[width=0.8\textwidth]{plots/carla_maps_used.png}
% \caption{Overview of CARLA maps used in the simulation study: Town01, Town02, Town03, Town04, Town05, and Town07. These maps provide diverse urban driving environments with varying road layouts, intersections, and traffic patterns.}
% \label{fig:carla_maps}
% \end{figure}

A script has been generated to simulate multiple vehicle types including trucks, motorcycles, 
and regular cars with varying probabilities, each exhibiting different behavioral
characteristics such as speed preferences, following distances, and lane-changing tendencies. 
The script incorporates dynamic behavior modifications during simulation, including random slowdowns, 
periodic behavior changes, and adaptive responses to traffic conditions, resulting in rich and 
varied traffic scene data across multiple CARLA maps and simulation iterations. The simulation runs
have between 20 and 60 vehicles each.

The resulting data consists of 2050 scenes with 11 seconds of simulation time each. The actors are spread over the entire map, 
so one scene produces multiple connected components, if the actor groups are far apart from each other.

\subsection{Map Graph Construction}
\label{subsec:map_graph}

The map graph is a directed multigraph $G_{\text{map}} = (V_{\text{map}}, E_{\text{map}})$ where each node $v \in V_{\text{map}}$ represents a lane segment, storing geometric information (boundaries, centerline, length) and semantic attributes (intersection status, road and lane type). Edges encode three spatial relationships between lanes:
\textbf{following edges} connect lanes forming a continuous path in the same direction,
\textbf{neighbor edges} connect adjacent lanes traveling in the same direction, and
\textbf{opposite edges} connect lanes traveling in opposite directions.
The map graph is constructed by processing map data from Argoverse or CARLA to extract these relationships. Lanes with geometric overlap are marked as intersection lanes.

\subsection{Actor Graph Construction}
\label{subsec:actor_graph}

The actor graph $G_{\text{actor}} = (V_{\text{actor}}, E_{\text{actor}})$ represents dynamic relationships between actors at a specific timestep. Each node $v \in V_{\text{actor}}$ represents an actor and stores attributes including primary lane ID, 
all occupied lane IDs, longitudinal position $s$ along the lane centerline, 3D position, longitudinal speed, actor type, and a lane change indicator. Each edge $e \in E_{\text{actor}}$ stores a actor-actor relation type and a path length along the lane network.

\label{subsubsec:relation_types}

Four relationship types are defined between actors, ordered by semantic hierarchy:
\begin{enumerate}
    \item \textbf{Following/Leading}: Longitudinal relationships where actors share the same lane or are connected entirely by following edges.
    \item \textbf{Neighbor}: Lateral relationships via paths containing exactly one neighbor edge (actors need not be on immediately adjacent lanes).
    \item \textbf{Opposite}: Relationships via paths containing exactly one opposite edge (actors need not be on immediately opposite lanes).
\end{enumerate}

% \begin{figure}[ht]
%       \centering
%       \begin{subfigure}[b]{0.48\textwidth}
%           \centering
%           \includegraphics[width=\textwidth]{plots/actor_map_graph/0922f82c-0640-43a6-b5ce-c42bc729418e_map_graph.png}
%           \caption{Map graph.}
%           \label{fig:map_graph_representation}
%       \end{subfigure}
%       \hfill
%       \begin{subfigure}[b]{0.48\textwidth}
%           \centering
%           \includegraphics[width=\textwidth]{plots/actor_map_graph/0922f82c-0640-43a6-b5ce-c42bc729418e_scene_at_1.0.png}
%           \caption{Traffic scene graph at one timestep.}
%           \label{fig:traffic_scene_timestep}
%       \end{subfigure}
%       \caption{The map graph (left) encodes spatial relationships between lanes. The traffic scene graph (right) shows actors and their relationships. Actors can be disconnected if they exceed the distance thresholds.}
%       \label{fig:map_and_scene_graphs}
%   \end{figure}

\subsection{Hierarchical Graph Construction Algorithm}
\label{subsec:construction_algorithm}

The actor graph is constructed in two phases that separate relation discovery from graph building, enabling hierarchical processing and preventing redundant edges. All parameters are listed in Table~\ref{tab:construction_parameters}.

\begin{table}[ht]
      \centering
      \caption{Input parameters for actor graph construction}
      \label{tab:construction_parameters}
      \begin{tabular}{llp{6cm}l}
      \hline
      \textbf{Parameter} & \textbf{Type} & \textbf{Description} & \textbf{Value} \\
      \hline
      \multicolumn{4}{l}{\textit{Distance limits (discovery phase)}} \\
      \hline
      \texttt{max\_distance\_lead\_veh\_m} & float & Max distance for leading/following & 100 \\
      \texttt{max\_distance\_neighbor\_fwd\_m} & float & Max distance for forward neighbor & 50 \\
      \texttt{max\_distance\_neighbor\_bwd\_m} & float & Max distance for backward neighbor & 50 \\
      \texttt{max\_distance\_opposite\_fwd\_m} & float & Max distance for forward opposite & 100 \\
      \texttt{max\_distance\_opposite\_bwd\_m} & float & Max distance for backward opposite & 10 \\
      \hline
      \multicolumn{4}{l}{\textit{Node distance limits (construction phase)}} \\
      \hline
      \texttt{max\_node\_dist\_leading} & int & Max edge count in path for leading/following & 3 \\
      \texttt{max\_node\_dist\_neighbor} & int & Max edge count in path for neighbor & 2 \\
      \texttt{max\_node\_dist\_opposite} & int & Max edge count in path for opposite & 2 \\
      \hline
      \multicolumn{4}{l}{\textit{Timestep configuration}} \\
      \hline
      \texttt{delta\_timestep\_s} & float & Time step increment in seconds & 1.0 \\
      \hline
      \end{tabular}
      \end{table}
% \subsubsection{Phase 1: Relation Discovery}
% \label{subsubsec:discovery_phase}
\textbf{Phase 1: Relation Discovery}: For each actor pair $(A, B)$ at timestep $t$, the algorithm determines their primary lanes and checks if a connecting path exists in the map graph. The path structure determines the relationship type: paths consisting entirely of following edges yield following/leading relations, paths with exactly one neighbor (or opposite) edge yield neighbor (or opposite) relations. Both lane-based path length and Euclidean distance are checked against the thresholds in Table~\ref{tab:construction_parameters} to filter distant actors.

\textbf{Phase 2: Hierarchical Graph Construction}: Edges are added in hierarchical order---leading/following first (highest priority), then neighbor, then opposite---each sorted by path length (shortest first). Before adding an edge between actors $A$ and $B$, a breadth-first search checks whether a path of length $\leq \text{max\_node\_distance}$ already exists in the current graph. If so, the direct edge is skipped, as the relationship is already implicitly encoded. The graph is updated after each edge addition so that subsequent checks reflect the current state.

This redundancy prevention significantly reduces edges while preserving connectivity. For example, three vehicles in a row need only two lead-follow edges; the third pairwise relation is encoded through the intermediate vehicle. Similarly, for a row of vehicles adjacent to an opposite-direction vehicle, only the closest pairwise relation needs an explicit edge (see Figure~\ref{fig:graph_construction_comparison}).

\begin{figure}[ht]
    \centering
    \begin{subfigure}[b]{0.48\textwidth}
        \centering
        \includegraphics[width=\textwidth]{plots/graph_construction/5758074f-be16-49da-8bb5-d43a0d8cd034_after_discovery.png}
        \caption{After relation discovery}
        \label{fig:graph_after_discovery}
    \end{subfigure}
    \hfill
    \begin{subfigure}[b]{0.48\textwidth}
        \centering
        \includegraphics[width=\textwidth]{plots/graph_construction/5758074f-be16-49da-8bb5-d43a0d8cd034_final.png}
        \caption{After hierarchical selection}
        \label{fig:graph_final}
    \end{subfigure}
    \caption{The discovery graph (left) contains all relations within distance limits. Hierarchical selection (right) removes redundant edges representable through existing paths.}
    \label{fig:graph_construction_comparison}
\end{figure}

