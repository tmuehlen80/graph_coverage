\section{Summary}

This paper presented a graph-based framework for coverage analysis in autonomous driving that represents traffic scenes as hierarchical graphs combining map topology with actor relationships.
Two complementary analysis approaches were developed: subgraph isomorphism using manually defined archetype graphs for interpretable pattern-based coverage metrics, 
and graph embeddings via GINE networks trained with contrastive learning for similarity-based assessment, clustering, and anomaly detection.

<<<<<<< HEAD
Validation on Argoverse 2.0 and CARLA data demonstrates that the graph representations capture complex traffic situations including lane changes, intersections, and multi-actor interactions (Section~\ref{chapter:application}).
The coverage analysis reveals significant distribution differences between real-world and simulated datasets.
Compared to approaches like TNO Streetwise \cite{tno_streetwise}, the framework scales efficiently without scenario-specific handling rules and naturally accommodates varying numbers of actors.
=======
Both approaches were validated on Argoverse 2.0 and CARLA data. It was shown that hierarichal graphs capture complex traffic situtations efficiently and allow the user to define individual specifics on defintions like follow vehicle, 
opposing vehicle and the set of relevant parameters.

We demonstrated that the graphs capture meaningful traffic scene structure and reveal differences between two datasets (Section~\ref{chapter:application}). First, the bottom-up subgraph approach revealed differences in traffic scnes, combination of 
traffic scenes and distribution of scenario parameters. Second, the top-down graph embedding approach revealed differences in the embedding space, allowing for further analysis without manual definition of archetypes. This representation 
can be the foundation for further analysis like similarity search, clustering and anomaly detection.

Compared to approaches like TNO Streetwise \cite{tno_streetwise}, the embedding approach scales efficiently without scenario-specific handling rules and naturally accommodates varying numbers of actors.
>>>>>>> mb/save_work

Future work will incorporate temporal information through multi-timestep graph structures and investigate automatic archetype extraction from real-world data.
% Also: better results when creating embeddings for subset of ODD

The source code is publicly available at \url{https://github.com/tmuehlen80/graph_coverage}.

