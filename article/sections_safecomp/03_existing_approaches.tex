\section{Existing coverage and analysis approaches}
\label{chapter:existing_approaches}

Foundational terminology for automated driving is established by the SAE J3016 taxonomy \cite{ORAD2021taxonomy}, which defines six levels of driving automation and key concepts such as the Dynamic Driving Task (DDT) and Operational Design Domain (ODD). Building on this, \cite{Ulbrich2015scene} provide precise definitions for the commonly conflated terms \emph{scene}, \emph{situation}, and \emph{scenario}, enabling a consistent vocabulary for test design and coverage arguments.

Scenario-based testing has emerged as the dominant paradigm for validating automated driving functions. The PEGASUS project \cite{pegasus2019method} introduces a six-layer model to decompose the driving environment into structured logical scenarios, shifting validation from infeasible distance-based testing toward systematic scenario exploration across simulation and field tests. \cite{DBLP:journals/corr/abs-1801-08598} refine this with three abstraction levels---functional, logical, and concrete scenarios---while noting that existing parameter-selection methods lack a systematic way to determine meaningful test coverage. Complementing these frameworks, \cite{deGelder2022ontology} propose object-oriented ontologies for scenario description with explicit linkage to coverage arguments, and \cite{Ries2021traj_clustering} present trajectory-based clustering of real-world driving data to structure the scenario space and reduce test-set redundancy.

A central challenge is the \emph{approval trap} identified by \cite{wachenfeld2016release}: statistically demonstrating superior safety through real-world driving alone would require billions of test kilometres, motivating the need for simulation-based tools and systematic coverage methods. General software testing theory \cite{Ammann_Offutt_2008} provides foundational coverage concepts (statement, branch, condition coverage), while \cite{foretellix2019} demonstrate their domain-specific application through coverage buckets, parameterised items, and performance metrics such as time-to-collision. At the standards level, ISO~21448 \cite{iso21448} and UL~4600 \cite{ul4600} both define frameworks for coverage criteria and measurement in the context of automated driving safety.

