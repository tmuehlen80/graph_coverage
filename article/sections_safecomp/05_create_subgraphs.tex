\section{Create subgraphs for coverage analysis}
\label{chapter:create_subgraphs_for_coverage_analysis}

Traffic scene archetypes from the literature (e.g., lead vehicle following, opposite traffic) can be expressed as small graphs. To determine whether such an archetype appears in a 
larger traffic scene graph $G$, we check if any subgraph of $G$ is isomorphic to the archetype graph $A$. While subgraph isomorphism is NP-hard in general, 
the graphs considered here are small enough for practical computation using the VF2 algorithm~\cite{cordella2004subgraph}.

Our strategy is as follows: (1)~Define a set of archetype subgraphs $S$ representing relevant traffic situations. (2)~Specify which node and edge attributes are considered for the isomorphism check. (3)~For each traffic scene graph $G$ (e.g., from CARLA or Argoverse), check if any subgraph of $G$ is isomorphic to any archetype in $S$ and record the result in a coverage table $C$.

This bottom-up approach starts from individual situation archetypes and checks their presence across datasets. The resulting coverage table enables follow-up analyses such as visualizing attribute distributions (speed, distance) for matched scenes, cross-tabulating co-occurring archetypes, or computing AV performance metrics conditioned on specific archetypes.

We defined 18 subgraph archetypes covering common traffic scenarios: simple 2-actor patterns (following, opposite, neighbor---used only for isolated pairs), 3-actor patterns (lead vehicle with neighbor, platoons, opposite traffic, lane change/cut-in scenarios with intersection variants), 
4-actor patterns (cut-out, multi-vehicle platoons, lead-follow with opposite traffic and intersection variants), and 5-actor patterns combining lead, neighbor, and opposite vehicles with intersection variants. The last step in the analysis of the subgraphs is to identify coverage differences between real-world and simulation datasets. The archetypes provide an intuitive understanding of what traffic situations are missing from the simulated environment. 

These concepts will be applied to CARLA and Argoverse datasets in the application chapter \ref{chapter:application}.